\documentclass[ignorenonframetext,]{beamer}
\setbeamertemplate{caption}[numbered]
\setbeamertemplate{caption label separator}{: }
\setbeamercolor{caption name}{fg=normal text.fg}
\beamertemplatenavigationsymbolsempty
\usepackage{lmodern}
\usepackage{amssymb,amsmath}
\usepackage{ifxetex,ifluatex}
\usepackage{fixltx2e} % provides \textsubscript
\ifnum 0\ifxetex 1\fi\ifluatex 1\fi=0 % if pdftex
  \usepackage[T1]{fontenc}
  \usepackage[utf8]{inputenc}
\else % if luatex or xelatex
  \ifxetex
    \usepackage{mathspec}
  \else
    \usepackage{fontspec}
  \fi
  \defaultfontfeatures{Ligatures=TeX,Scale=MatchLowercase}
\fi
% use upquote if available, for straight quotes in verbatim environments
\IfFileExists{upquote.sty}{\usepackage{upquote}}{}
% use microtype if available
\IfFileExists{microtype.sty}{%
\usepackage{microtype}
\UseMicrotypeSet[protrusion]{basicmath} % disable protrusion for tt fonts
}{}
\newif\ifbibliography
\hypersetup{
            pdftitle={An Introduction to Git, RStudio, and R Markdown},
            pdfauthor={Bayesian Psychometric Models, Lecture 2},
            pdfborder={0 0 0},
            breaklinks=true}
\urlstyle{same}  % don't use monospace font for urls
\usepackage{color}
\usepackage{fancyvrb}
\newcommand{\VerbBar}{|}
\newcommand{\VERB}{\Verb[commandchars=\\\{\}]}
\DefineVerbatimEnvironment{Highlighting}{Verbatim}{commandchars=\\\{\}}
% Add ',fontsize=\small' for more characters per line
\usepackage{framed}
\definecolor{shadecolor}{RGB}{248,248,248}
\newenvironment{Shaded}{\begin{snugshade}}{\end{snugshade}}
\newcommand{\KeywordTok}[1]{\textcolor[rgb]{0.13,0.29,0.53}{\textbf{#1}}}
\newcommand{\DataTypeTok}[1]{\textcolor[rgb]{0.13,0.29,0.53}{#1}}
\newcommand{\DecValTok}[1]{\textcolor[rgb]{0.00,0.00,0.81}{#1}}
\newcommand{\BaseNTok}[1]{\textcolor[rgb]{0.00,0.00,0.81}{#1}}
\newcommand{\FloatTok}[1]{\textcolor[rgb]{0.00,0.00,0.81}{#1}}
\newcommand{\ConstantTok}[1]{\textcolor[rgb]{0.00,0.00,0.00}{#1}}
\newcommand{\CharTok}[1]{\textcolor[rgb]{0.31,0.60,0.02}{#1}}
\newcommand{\SpecialCharTok}[1]{\textcolor[rgb]{0.00,0.00,0.00}{#1}}
\newcommand{\StringTok}[1]{\textcolor[rgb]{0.31,0.60,0.02}{#1}}
\newcommand{\VerbatimStringTok}[1]{\textcolor[rgb]{0.31,0.60,0.02}{#1}}
\newcommand{\SpecialStringTok}[1]{\textcolor[rgb]{0.31,0.60,0.02}{#1}}
\newcommand{\ImportTok}[1]{#1}
\newcommand{\CommentTok}[1]{\textcolor[rgb]{0.56,0.35,0.01}{\textit{#1}}}
\newcommand{\DocumentationTok}[1]{\textcolor[rgb]{0.56,0.35,0.01}{\textbf{\textit{#1}}}}
\newcommand{\AnnotationTok}[1]{\textcolor[rgb]{0.56,0.35,0.01}{\textbf{\textit{#1}}}}
\newcommand{\CommentVarTok}[1]{\textcolor[rgb]{0.56,0.35,0.01}{\textbf{\textit{#1}}}}
\newcommand{\OtherTok}[1]{\textcolor[rgb]{0.56,0.35,0.01}{#1}}
\newcommand{\FunctionTok}[1]{\textcolor[rgb]{0.00,0.00,0.00}{#1}}
\newcommand{\VariableTok}[1]{\textcolor[rgb]{0.00,0.00,0.00}{#1}}
\newcommand{\ControlFlowTok}[1]{\textcolor[rgb]{0.13,0.29,0.53}{\textbf{#1}}}
\newcommand{\OperatorTok}[1]{\textcolor[rgb]{0.81,0.36,0.00}{\textbf{#1}}}
\newcommand{\BuiltInTok}[1]{#1}
\newcommand{\ExtensionTok}[1]{#1}
\newcommand{\PreprocessorTok}[1]{\textcolor[rgb]{0.56,0.35,0.01}{\textit{#1}}}
\newcommand{\AttributeTok}[1]{\textcolor[rgb]{0.77,0.63,0.00}{#1}}
\newcommand{\RegionMarkerTok}[1]{#1}
\newcommand{\InformationTok}[1]{\textcolor[rgb]{0.56,0.35,0.01}{\textbf{\textit{#1}}}}
\newcommand{\WarningTok}[1]{\textcolor[rgb]{0.56,0.35,0.01}{\textbf{\textit{#1}}}}
\newcommand{\AlertTok}[1]{\textcolor[rgb]{0.94,0.16,0.16}{#1}}
\newcommand{\ErrorTok}[1]{\textcolor[rgb]{0.64,0.00,0.00}{\textbf{#1}}}
\newcommand{\NormalTok}[1]{#1}
\usepackage{graphicx,grffile}
\makeatletter
\def\maxwidth{\ifdim\Gin@nat@width>\linewidth\linewidth\else\Gin@nat@width\fi}
\def\maxheight{\ifdim\Gin@nat@height>\textheight0.8\textheight\else\Gin@nat@height\fi}
\makeatother
% Scale images if necessary, so that they will not overflow the page
% margins by default, and it is still possible to overwrite the defaults
% using explicit options in \includegraphics[width, height, ...]{}
\setkeys{Gin}{width=\maxwidth,height=\maxheight,keepaspectratio}

% Prevent slide breaks in the middle of a paragraph:
\widowpenalties 1 10000
\raggedbottom

\AtBeginPart{
  \let\insertpartnumber\relax
  \let\partname\relax
  \frame{\partpage}
}
\AtBeginSection{
  \ifbibliography
  \else
    \let\insertsectionnumber\relax
    \let\sectionname\relax
    \frame{\sectionpage}
  \fi
}
\AtBeginSubsection{
  \let\insertsubsectionnumber\relax
  \let\subsectionname\relax
  \frame{\subsectionpage}
}

\setlength{\parindent}{0pt}
\setlength{\parskip}{6pt plus 2pt minus 1pt}
\setlength{\emergencystretch}{3em}  % prevent overfull lines
\providecommand{\tightlist}{%
  \setlength{\itemsep}{0pt}\setlength{\parskip}{0pt}}
\setcounter{secnumdepth}{0}

\title{An Introduction to Git, RStudio, and R Markdown}
\author{Bayesian Psychometric Models, Lecture 2}
\date{}

\begin{document}
\frame{\titlepage}

\begin{frame}

\end{frame}

\begin{frame}{In This Lecture:}

\begin{enumerate}
\def\labelenumi{\arabic{enumi}.}
\tightlist
\item
  Git (and how to access the course GitHub repo)
\item
  RStudio (and how to use Git within RStudio)
\item
  R Markdown (and how to make it run to play with syntax during class)
\end{enumerate}

\end{frame}

\begin{frame}{Git Clients}

\begin{itemize}
\tightlist
\item
  If you use a Mac/Linux machine, you already have a client as part of
  the OS.
\item
  If you use Windows, you will have to download and install a client to
  use Git

  \begin{itemize}
  \tightlist
  \item
    A good client to use with RStudio is
    \url{https://git-scm.com/download/win}
  \end{itemize}
\end{itemize}

\end{frame}

\begin{frame}{The Basics of Git}

\begin{itemize}
\tightlist
\item
  Git is a version control system that helps to keep track of changes to
  files across the lifespan of a project

  \begin{itemize}
  \tightlist
  \item
    It is also great for using in collaboration with others
  \end{itemize}
\item
  An overview of Git is given in many places here are a few:

  \begin{itemize}
  \tightlist
  \item
    \url{http://rogerdudler.github.io/git-guide/} (thank you,
    Dr.~LeBeau)
  \item
    \url{https://git-scm.com/book/en/v2/Getting-Started-Git-Basics}
  \end{itemize}
\item
  I will use these websites to help introduce you to Git concepts
\item
  Also, I will highlight the GitHub site with our course repo:
  \url{https://github.com/jonathantemplin/BayesianPsychometricModeling}
\end{itemize}

\end{frame}

\begin{frame}{Using Git with R Studio}

\begin{itemize}
\tightlist
\item
  Git makes our life easy when used with RStudio as we can use it to
  keep a current copy of our course notes available

  \begin{itemize}
  \tightlist
  \item
    We will use Git to download and update our course materials in
    RStudio
  \end{itemize}
\item
  Do do this:

  \begin{enumerate}
  \def\labelenumi{\arabic{enumi}.}
  \tightlist
  \item
    Open RStudio
  \item
    At the menu on the top, go to File\ldots{}New Project
  \item
    Select ``Version Control'' on the ``Create Project'' window
  \item
    Select ``Git'' on the ``Create Project from Version Control'' window
  \item
    On the ``Clone Git Repository'' window input:

    \begin{itemize}
    \tightlist
    \item
      Repository URL:
      \url{https://github.com/jonathantemplin/BayesianPsychometricModeling}
    \item
      Project directory name: Choose a directory name for the course
      materials (such as ``BPM Git Repo'')

      \begin{itemize}
      \tightlist
      \item
        NOTE: The master branch has an up-to-date R project file
        (.Rproj) that will have all files included.
      \end{itemize}
    \item
      Create project as a subdirectory of: Choose a location for your
      files to reside within on your local machine
    \end{itemize}
  \end{enumerate}
\end{itemize}

\end{frame}

\begin{frame}{Using RStudio}

\begin{itemize}
\tightlist
\item
  Next I will demonstrate RStudio for you using the contents of last
  week's R script file
\item
  For more information about the RStudio Integrated Development
  Environment (IDE), see the following links:

  \begin{itemize}
  \tightlist
  \item
    \url{https://www.rstudio.com/online-learning/}
  \item
    \url{https://dss.princeton.edu/training/RStudio101.pdf}
  \end{itemize}
\item
  Also, to really unlock RStudio's full potential, familiarize yourself
  with its keyboard shortcuts:

  \begin{itemize}
  \tightlist
  \item
    In the top menu, go to Tools\ldots{}Keyboard Shortcuts Help
  \end{itemize}
\end{itemize}

\end{frame}

\begin{frame}{Using R Markdown}

\begin{itemize}
\item
  Last week was the exception in that I did not provide course materials
  in R Markdown\ldots{}today that changes
\item
  R Markdown is a form of the Markdown language
  (\url{https://en.wikipedia.org/wiki/Markdown})

  \begin{itemize}
  \tightlist
  \item
    Markdown was the counter to HTML (Hypertext Markup
    Language)\ldots{}but is a markup language that is very easy to use
  \item
    You can find Markdown nearly everywhere these days (see your
    Notes/OneNote application)
  \end{itemize}
\item
  Markdown makes writing very simple:

  \begin{itemize}
  \tightlist
  \item
    It works nearly everywhere (files are simple text)
  \item
    It can incorporate more complicated markup languages, such as LaTex:
    \(P \left(\theta|Y \right) \propto P \left(Y|\theta \right) P \left( \theta \right)\)
  \end{itemize}
\item
  When you do need a type of document, you can then use any number of
  programs to make it look nice:

  \begin{itemize}
  \tightlist
  \item
    Pandoc (\url{https://pandoc.org}; converts to Word, PDF, LaTex,
    etc\ldots{})
  \item
    The papaja R package (\url{https://crsh.github.io/papaja_man/})
  \end{itemize}
\item
  R Markdown allows you to embed R script within the document, providing
  syntax snippets and output directly to your final document format
\item
  You can find lots of helpful tips on R Markdown on some of these
  sites:

  \begin{itemize}
  \tightlist
  \item
    \url{https://rmarkdown.rstudio.com}
  \item
    \url{https://www.rstudio.com/wp-content/uploads/2015/02/rmarkdown-cheatsheet.pdf}
  \end{itemize}
\end{itemize}

\end{frame}

\begin{frame}[fragile]{R Markdown Example}

\begin{itemize}
\tightlist
\item
  Recalling last week's R script for theta, below is how to embed it in
  R Markdown

  \begin{itemize}
  \tightlist
  \item
    Note: I've changed the number of iterations to something very small
    to make it run fast
  \end{itemize}
\end{itemize}

\begin{Shaded}
\begin{Highlighting}[]
\NormalTok{irtItemProb =}\StringTok{ }\ControlFlowTok{function}\NormalTok{(a, b, }\DataTypeTok{c=}\DecValTok{0}\NormalTok{, theta)\{}
\NormalTok{  prob =}\StringTok{ }\NormalTok{c }\OperatorTok{+}\StringTok{ }\NormalTok{(}\DecValTok{1}\OperatorTok{-}\NormalTok{c) }\OperatorTok{*}\StringTok{ }\KeywordTok{exp}\NormalTok{(a}\OperatorTok{*}\NormalTok{(theta}\OperatorTok{-}\NormalTok{b))}\OperatorTok{/}\NormalTok{(}\DecValTok{1}\OperatorTok{+}\KeywordTok{exp}\NormalTok{(a}\OperatorTok{*}\NormalTok{(theta}\OperatorTok{-}\NormalTok{b)))}
  \KeywordTok{return}\NormalTok{(prob)}
\NormalTok{\}}

\NormalTok{trueTheta =}\StringTok{ }\DecValTok{0}
\NormalTok{nItems =}\StringTok{ }\DecValTok{5}
\NormalTok{nItems =}\StringTok{ }\DecValTok{5}
\NormalTok{bRange =}\StringTok{ }\KeywordTok{c}\NormalTok{(}\OperatorTok{-}\DecValTok{2}\NormalTok{,}\DecValTok{2}\NormalTok{)}
\NormalTok{aRange =}\StringTok{ }\KeywordTok{c}\NormalTok{(}\DecValTok{1}\NormalTok{,}\DecValTok{2}\NormalTok{)}
\NormalTok{bSE =}\StringTok{ }\DecValTok{1}
\NormalTok{aSE =}\StringTok{ }\DecValTok{1}
\NormalTok{nSamples =}\StringTok{ }\DecValTok{1000}

\CommentTok{# draw mean values of a, b}
\NormalTok{a =}\StringTok{ }\KeywordTok{runif}\NormalTok{(}\DataTypeTok{n =}\NormalTok{ nItems, }\DataTypeTok{min =}\NormalTok{ aRange[}\DecValTok{1}\NormalTok{], }\DataTypeTok{max =}\NormalTok{ aRange[}\DecValTok{2}\NormalTok{])}
\NormalTok{b =}\StringTok{ }\KeywordTok{runif}\NormalTok{(}\DataTypeTok{n =}\NormalTok{ nItems, }\DataTypeTok{min =}\NormalTok{ bRange[}\DecValTok{1}\NormalTok{], }\DataTypeTok{max =}\NormalTok{ bRange[}\DecValTok{2}\NormalTok{])}

\CommentTok{# draw items}
\NormalTok{itemResponses =}\StringTok{ }\KeywordTok{rbinom}\NormalTok{(}\DataTypeTok{n =}\NormalTok{ nItems, }\DataTypeTok{size =} \DecValTok{1}\NormalTok{, }\DataTypeTok{prob =} \KeywordTok{irtItemProb}\NormalTok{(}\DataTypeTok{a =}\NormalTok{ a, }\DataTypeTok{b =}\NormalTok{ b, }\DataTypeTok{theta =} \DecValTok{1}\NormalTok{))}
\NormalTok{thetaChain =}\StringTok{ }\KeywordTok{list}\NormalTok{(}\KeywordTok{rep}\NormalTok{(}\OtherTok{NA}\NormalTok{, nSamples), }\KeywordTok{rep}\NormalTok{(}\OtherTok{NA}\NormalTok{, nSamples))}

\CommentTok{# initialize theta values}
\NormalTok{curTheta =}\StringTok{ }\NormalTok{trueTheta}
\NormalTok{curThetaRand =}\StringTok{ }\NormalTok{trueTheta}
\ControlFlowTok{for}\NormalTok{ (iteration }\ControlFlowTok{in} \DecValTok{1}\OperatorTok{:}\NormalTok{nSamples)\{}
  
  \CommentTok{# draw item parameters (if random)}
\NormalTok{  iterA =}\StringTok{ }\KeywordTok{rnorm}\NormalTok{(}\DataTypeTok{n =}\NormalTok{ nItems, }\DataTypeTok{mean =}\NormalTok{ a, }\DataTypeTok{sd =}\NormalTok{ aSE)}
\NormalTok{  iterB =}\StringTok{ }\KeywordTok{rnorm}\NormalTok{(}\DataTypeTok{n =}\NormalTok{ nItems, }\DataTypeTok{mean =}\NormalTok{ b, }\DataTypeTok{sd =}\NormalTok{ bSE)}
  
  \CommentTok{# calculate current likelihood of the data | theta}
\NormalTok{  curLogLike =}\StringTok{ }\KeywordTok{sum}\NormalTok{(}\KeywordTok{dbinom}\NormalTok{(}\DataTypeTok{x =}\NormalTok{ itemResponses, }\DataTypeTok{size =} \DecValTok{1}\NormalTok{, }\DataTypeTok{prob =} \KeywordTok{irtItemProb}\NormalTok{(}\DataTypeTok{a =}\NormalTok{ a, }\DataTypeTok{b =}\NormalTok{ b, }\DataTypeTok{theta =}\NormalTok{ curTheta), }\DataTypeTok{log =} \OtherTok{TRUE}\NormalTok{))}
\NormalTok{  curLogLikeRand =}\StringTok{ }\KeywordTok{sum}\NormalTok{(}\KeywordTok{dbinom}\NormalTok{(}\DataTypeTok{x =}\NormalTok{ itemResponses, }\DataTypeTok{size =} \DecValTok{1}\NormalTok{, }\DataTypeTok{prob =} \KeywordTok{irtItemProb}\NormalTok{(}\DataTypeTok{a =}\NormalTok{ iterA, }\DataTypeTok{b =}\NormalTok{ iterB, }\DataTypeTok{theta =}\NormalTok{ curThetaRand), }\DataTypeTok{log =} \OtherTok{TRUE}\NormalTok{))}
  
  \CommentTok{# draw new theta value}
\NormalTok{  propTheta =}\StringTok{ }\KeywordTok{rnorm}\NormalTok{(}\DataTypeTok{n =} \DecValTok{1}\NormalTok{, }\DataTypeTok{mean =}\NormalTok{ curTheta, }\DataTypeTok{sd =} \DecValTok{1}\NormalTok{)}
\NormalTok{  propThetaRand =}\StringTok{ }\KeywordTok{rnorm}\NormalTok{(}\DataTypeTok{n =} \DecValTok{1}\NormalTok{, }\DataTypeTok{mean =}\NormalTok{ curThetaRand, }\DataTypeTok{sd =} \DecValTok{1}\NormalTok{)}
  
  \CommentTok{# calculate proposed likelihood of the data | theta}
\NormalTok{  propLogLike =}\StringTok{ }\KeywordTok{sum}\NormalTok{(}\KeywordTok{dbinom}\NormalTok{(}\DataTypeTok{x =}\NormalTok{ itemResponses, }\DataTypeTok{size =} \DecValTok{1}\NormalTok{, }\DataTypeTok{prob =} \KeywordTok{irtItemProb}\NormalTok{(}\DataTypeTok{a =}\NormalTok{ a, }\DataTypeTok{b =}\NormalTok{ b, }\DataTypeTok{theta =}\NormalTok{ propTheta), }\DataTypeTok{log =} \OtherTok{TRUE}\NormalTok{))}
\NormalTok{  propLogLikeRand =}\StringTok{ }\KeywordTok{sum}\NormalTok{(}\KeywordTok{dbinom}\NormalTok{(}\DataTypeTok{x =}\NormalTok{ itemResponses, }\DataTypeTok{size =} \DecValTok{1}\NormalTok{, }\DataTypeTok{prob =} \KeywordTok{irtItemProb}\NormalTok{(}\DataTypeTok{a =}\NormalTok{ iterA, }\DataTypeTok{b =}\NormalTok{ iterB, }\DataTypeTok{theta =}\NormalTok{ propThetaRand), }\DataTypeTok{log =} \OtherTok{TRUE}\NormalTok{))}
  
  \CommentTok{# do MH:}
  \ControlFlowTok{if}\NormalTok{ (}\KeywordTok{log}\NormalTok{(}\KeywordTok{runif}\NormalTok{(}\DataTypeTok{n =} \DecValTok{1}\NormalTok{)) }\OperatorTok{<}\StringTok{ }\NormalTok{(propLogLike}\OperatorTok{-}\NormalTok{curLogLike))\{}
    \CommentTok{# accept}
\NormalTok{    curTheta =}\StringTok{ }\NormalTok{propTheta}
\NormalTok{  \} }

  \CommentTok{# do MH:}
  \ControlFlowTok{if}\NormalTok{ (}\KeywordTok{log}\NormalTok{(}\KeywordTok{runif}\NormalTok{(}\DataTypeTok{n =} \DecValTok{1}\NormalTok{)) }\OperatorTok{<}\StringTok{ }\NormalTok{(propLogLikeRand}\OperatorTok{-}\NormalTok{curLogLikeRand))\{}
    \CommentTok{# accept}
\NormalTok{    curThetaRand =}\StringTok{ }\NormalTok{propThetaRand}
\NormalTok{  \}}
  
\NormalTok{  thetaChain[[}\DecValTok{1}\NormalTok{]][iteration] =}\StringTok{ }\NormalTok{curTheta}
\NormalTok{  thetaChain[[}\DecValTok{2}\NormalTok{]][iteration] =}\StringTok{ }\NormalTok{curThetaRand}
\NormalTok{\}}

\KeywordTok{par}\NormalTok{(}\DataTypeTok{mfrow =} \KeywordTok{c}\NormalTok{(}\DecValTok{1}\NormalTok{,}\DecValTok{2}\NormalTok{))}

\KeywordTok{plot}\NormalTok{(thetaChain[[}\DecValTok{1}\NormalTok{]], }\DataTypeTok{type=}\StringTok{"l"}\NormalTok{, }\DataTypeTok{ylab =} \KeywordTok{expression}\NormalTok{(theta), }\DataTypeTok{xlab =} \StringTok{"Iteration Number"}\NormalTok{)}
\KeywordTok{lines}\NormalTok{(thetaChain[[}\DecValTok{2}\NormalTok{]], }\DataTypeTok{type=}\StringTok{"l"}\NormalTok{, }\DataTypeTok{col =} \DecValTok{2}\NormalTok{)}
\KeywordTok{plot}\NormalTok{(}\KeywordTok{density}\NormalTok{(thetaChain[[}\DecValTok{1}\NormalTok{]]), }\DataTypeTok{col =} \DecValTok{1}\NormalTok{, }\DataTypeTok{main=}\StringTok{""}\NormalTok{)}
\KeywordTok{lines}\NormalTok{(}\KeywordTok{density}\NormalTok{(thetaChain[[}\DecValTok{2}\NormalTok{]]), }\DataTypeTok{col =} \DecValTok{2}\NormalTok{)}
\end{Highlighting}
\end{Shaded}

\includegraphics{bpm19psqf7375_Lecture02_files/figure-beamer/unnamed-chunk-1-1.pdf}

\begin{Shaded}
\begin{Highlighting}[]
\KeywordTok{par}\NormalTok{(}\DataTypeTok{mfrow =} \KeywordTok{c}\NormalTok{(}\DecValTok{3}\NormalTok{,}\DecValTok{2}\NormalTok{))}
\KeywordTok{plot}\NormalTok{(thetaChain[[}\DecValTok{1}\NormalTok{]], }\DataTypeTok{type=}\StringTok{"l"}\NormalTok{, }\DataTypeTok{ylab =} \KeywordTok{expression}\NormalTok{(theta), }\DataTypeTok{xlab =} \StringTok{"Iteration Number"}\NormalTok{)}
\KeywordTok{plot}\NormalTok{(thetaChain[[}\DecValTok{2}\NormalTok{]], }\DataTypeTok{type=}\StringTok{"l"}\NormalTok{, }\DataTypeTok{ylab =} \KeywordTok{expression}\NormalTok{(theta), }\DataTypeTok{xlab =} \StringTok{"Iteration Number"}\NormalTok{, }\DataTypeTok{col =}\DecValTok{2}\NormalTok{)}
\KeywordTok{plot}\NormalTok{(}\KeywordTok{density}\NormalTok{(thetaChain[[}\DecValTok{1}\NormalTok{]]), }\DataTypeTok{col =} \DecValTok{1}\NormalTok{, }\DataTypeTok{main=}\StringTok{""}\NormalTok{)}
\KeywordTok{plot}\NormalTok{(}\KeywordTok{density}\NormalTok{(thetaChain[[}\DecValTok{2}\NormalTok{]]), }\DataTypeTok{col =} \DecValTok{2}\NormalTok{, }\DataTypeTok{main=}\StringTok{""}\NormalTok{)}
\KeywordTok{plot}\NormalTok{(thetaChain[[}\DecValTok{1}\NormalTok{]], }\DataTypeTok{type=}\StringTok{"l"}\NormalTok{, }\DataTypeTok{ylab =} \KeywordTok{expression}\NormalTok{(theta), }\DataTypeTok{xlab =} \StringTok{"Iteration Number"}\NormalTok{)}
\KeywordTok{lines}\NormalTok{(thetaChain[[}\DecValTok{2}\NormalTok{]], }\DataTypeTok{type=}\StringTok{"l"}\NormalTok{, }\DataTypeTok{col =} \DecValTok{2}\NormalTok{)}
\KeywordTok{plot}\NormalTok{(}\KeywordTok{density}\NormalTok{(thetaChain[[}\DecValTok{1}\NormalTok{]]), }\DataTypeTok{col =} \DecValTok{1}\NormalTok{, }\DataTypeTok{main=}\StringTok{""}\NormalTok{)}
\KeywordTok{lines}\NormalTok{(}\KeywordTok{density}\NormalTok{(thetaChain[[}\DecValTok{2}\NormalTok{]]), }\DataTypeTok{col =} \DecValTok{2}\NormalTok{)}
\end{Highlighting}
\end{Shaded}

\includegraphics{bpm19psqf7375_Lecture02_files/figure-beamer/unnamed-chunk-1-2.pdf}

\end{frame}

\begin{frame}{More R Markdown}

\begin{itemize}
\item
  To compile the whole document (called ``Knitting'' as it uses a
  package named knitr), press the Knit button or use the keystroke
  command-shift-K
\item
  To run a chunk of R code, find and press the button on the top right
  of the chunk
\item
  Note: If chunks later in the document depend on chunks at the
  beginning, you will have to run the beginning ones first (chunks use
  the current Global R Environment for variables and functions)
\end{itemize}

\end{frame}

\end{document}
